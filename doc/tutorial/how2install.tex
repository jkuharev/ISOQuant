\documentclass[]{article}
\usepackage[T1]{fontenc}
\usepackage{lmodern}
\usepackage{amssymb,amsmath}
\usepackage{ifxetex,ifluatex}
\usepackage{fixltx2e} % provides \textsubscript
% use microtype if available
\IfFileExists{microtype.sty}{\usepackage{microtype}}{}
\ifnum 0\ifxetex 1\fi\ifluatex 1\fi=0 % if pdftex
  \usepackage[utf8]{inputenc}
\else % if luatex or xelatex
  \usepackage{fontspec}
  \ifxetex
    \usepackage{xltxtra,xunicode}
  \fi
  \defaultfontfeatures{Mapping=tex-text,Scale=MatchLowercase}
  \newcommand{\euro}{€}
\fi
\usepackage{listings}
\ifxetex
  \usepackage[setpagesize=false, % page size defined by xetex
              unicode=false, % unicode breaks when used with xetex
              xetex]{hyperref}
\else
  \usepackage[unicode=true]{hyperref}
\fi
\hypersetup{breaklinks=true,
            bookmarks=true,
            pdfauthor={Jörg Kuharev},
            pdftitle={ISOQuant installation guide},
            colorlinks=true,
            urlcolor=blue,
            linkcolor=magenta,
            pdfborder={0 0 0}}
\setlength{\parindent}{0pt}
\setlength{\parskip}{6pt plus 2pt minus 1pt}
\setlength{\emergencystretch}{3em}  % prevent overfull lines
\usepackage[left=2cm,right=1.5cm,top=1.5cm,bottom=1.5cm]{geometry}
% \setlength{\parindent}{0.25in}
\setlength{\parskip}{0.4cm}
\pagestyle{plain}

%\renewcommand{\figurename}{Abb.}
%\renewcommand{\tablename}{Tab.}
%\renewcommand{\listfigurename}{Abbildungsverzeichnis}
%\renewcommand{\listtablename}{Tabellenverzeichnis}
%\renewcommand{\refname}{Literaturverzeichnis}
\renewcommand{\contentsname}{Table of contents}

\renewcommand{\baselinestretch}{1.0}\normalsize

\title{ISOQuant installation guide}
\author{Jörg Kuharev}
\date{\today}

\begin{document}
\maketitle

{
\hypersetup{linkcolor=black}
\tableofcontents
}
\begin{center}\rule{3in}{0.4pt}\end{center}

\clearpage

\section{About installation guide}

This document describes how to install ISOQuant and related software.
Installation procedure may vary depending on your operating system
environment. This document only handles generic cases for Windows, Linux
and Mac OSX machines.

\section{Software requirements}

\begin{enumerate}
\item
  Java Virtual Machine (version 1.6 and newer or compatible) locally
  installed as the execution environment as ISOQuant is a Java
  application.
\item
  MySQL Server (version 5.1. and newer or compatible) locally installed
  or available on local network.
\item
  Waters ProteinLynx Global Server (PLGS) (version 2.3. or newer)\\ in
  fact, you don't need PLGS itself, however ISOQuant needs PLGS root
  folder to be locally available (as a local folder or as a locally
  mounted network path)
\item
  Web browser (Firefox, Safari, Opera, etc.) for viewing generated HTML
  reports
\item
  Spreadsheet viewer (MS-Excel or compatible, e.g.~LibreOffice Calc) for
  viewing generated XLS and CSV reports
\end{enumerate}

\section{Installation steps}

\subsection{Windows}

To simplify the installation process on Windows machines, we have
created a Windows installation package -
\lstinline!ISOQuant for Windows!. Packaged installation assistant will
guide you through the installation process.

\subsection{Mac OS X}

To simplify the installation process on Macintosh machines, we have
created a Mac OS X image file. The image file
\lstinline!ISOQuant for Mac! contains the ISOQuant.app, which should be
copied to a local folder, e.g. \lstinline!Applications! folder.

\subsection{Linux, Unix and other OS}

For advanced users, we have created a portable distribution package
\lstinline!ISOQuant portable!, which may be used on any operating
systems. The portable distribution is an archive containing all needed
files. Use packaged launcher scripts \lstinline!isoquant4unix.sh! or
\lstinline!isoquant4win.bat! to launch ISOQuant application.

\end{document}
